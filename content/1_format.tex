\section{Format}

The style file \Verb|document.sty|, written in \LaTeXe{}, provides a comprehensive document formatting solution.
The theorem-like environments are inspired by the template~\autocite{sleepymalc2022template}.
For detailed implementation, refer to the GitHub repository~\autocite{chang2025document}.

\subsection{Package Usage and Options}

To use this package, add \Verb|\usepackage{document}| after the \Verb|\documentclass| command.
The package accepts several key-value options:

\begin{description}
    \item[\texttt{theorems=true/false}] (default: \texttt{true}) Enable or disable theorem-like environments (definition, theorem, lemma, corollary, proof, example, etc.). When disabled, the \Verb|thmtools| and \Verb|tcolorbox| packages are not loaded.
    
    \item[\texttt{colors=true/false}] (default: \texttt{true}) Enable extended color support. When \texttt{true}, loads \Verb|xcolor| with \texttt{dvipsnames} option for additional color names.
    
    \item[\texttt{code=true/false}] (default: \texttt{true}) Enable syntax-highlighted code listings. When \texttt{true}, loads the \Verb|minted| package with pre-configured environments for LaTeX and Python code.
    
    \item[\texttt{draft=true/false}] (default: \texttt{false}) Enable draft mode for faster compilation (currently reserved for future use).
    
    \item[\texttt{timesfont=true/false}] (default: \texttt{true}) Use Times font family for the document text.
    
    \item[\texttt{header=default/formal}] (default: \texttt{default}) Select header style. The \texttt{default} style shows section titles at the center with page numbers in the footer. The \texttt{formal} style displays purpose, title, author, institute, and page count.
    
    \item[\texttt{bibstyle=style}] (default: \texttt{ieee}) Set bibliography style for \Verb|biblatex|. Common options include \texttt{ieee}, \texttt{apa}, \texttt{nature}, \texttt{numeric}, or \texttt{alphabetic}.
\end{description}

Example usage:
\begin{latexcode}
\usepackage[
  theorems=true,
  colors=true,
  code=true,
  header=formal,
  bibstyle=ieee
]{document}
\end{latexcode}

\subsection{Imported Packages}

The \Verb|document.sty| package imports and configures numerous packages organized by functionality:

\paragraph{Parsing and Options}
\begin{itemize}
    \item \Verb|xparse| --- Advanced command definition with flexible argument parsing
    \item \Verb|ifthen| --- Conditional statements and logic
    \item \Verb|etoolbox| --- Programming facilities and command patching
    \item \Verb|kvoptions| --- Key-value option handling for package options
\end{itemize}

\paragraph{Page Layout}
\begin{itemize}
    \item \Verb|geometry| --- Page dimensions and margins (0.5in sides, 1.5in bottom)
    \item \Verb|emptypage| --- Clean formatting for empty pages
    \item \Verb|fancyhdr| --- Customizable headers and footers
    \item \Verb|lastpage| --- Reference to the last page number (for formal header)
\end{itemize}

\paragraph{Text Encoding and Fonts}
\begin{itemize}
    \item \Verb|inputenc| (utf8) --- Unicode character input support
    \item \Verb|fontenc| (T1) --- Font encoding for proper character rendering
    \item \Verb|times| --- Times Roman font family (when \texttt{timesfont=true})
    \item \Verb|textcomp| --- Additional text symbols and companion fonts
    \item \Verb|fontawesome| --- Icon font support for symbols like \faGithub
\end{itemize}

\paragraph{Mathematics}
\begin{itemize}
    \item \Verb|amsmath| --- Enhanced mathematics environments and commands
    \item \Verb|amsfonts| --- Mathematical fonts including blackboard bold
    \item \Verb|amsthm| --- Theorem environment support
    \item \Verb|amssymb| --- Extended mathematical symbols
    \item \Verb|mathtools| --- Enhancements and fixes for \Verb|amsmath|
    \item \Verb|mathrsfs| --- Ralph Smith's Formal Script font for math
    \item \Verb|mathdots| --- Commands for improving the appearance of dots in math
    \item \Verb|nicefrac| --- Typeset nice diagonal fractions
    \item \Verb|cancel| --- Cross out mathematical expressions
    \item \Verb|MnSymbol| --- Additional mathematical symbols
    \item \Verb|dsfont| --- Doublestroke font for mathematical alphabets
    \item \Verb|upgreek| --- Upright Greek letters in mathematics
    \item \Verb|systeme| --- Systems of linear equations formatting
    \item \Verb|siunitx| --- Consistent SI units and number formatting
    \item \Verb|pifont| --- Access to Pi fonts for special symbols
\end{itemize}

\paragraph{Colors and Hyperlinks}
\begin{itemize}
    \item \Verb|xcolor| --- Color support with optional \texttt{dvipsnames} palette
    \item \Verb|hyperref| --- Hyperlinks and PDF metadata with color configuration
\end{itemize}

\paragraph{Text Formatting}
\begin{itemize}
    \item \Verb|soul| --- Letter spacing and highlighting (\Verb|\spaceout| command)
    \item \Verb|fancyvrb| --- Enhanced verbatim text for inline code
    \item \Verb|csquotes| --- Context-sensitive quotation marks
    \item \Verb|enumitem| --- Enhanced control over list environments
\end{itemize}

\paragraph{Code Listings}
\begin{itemize}
    \item \Verb|minted| --- Syntax highlighting using Pygments (when \texttt{code=true})
\end{itemize}

\paragraph{Tables}
\begin{itemize}
    \item \Verb|booktabs| --- Publication-quality table rules
    \item \Verb|arydshln| --- Dashed lines in tables
    \item \Verb|multirow| --- Cells spanning multiple rows
    \item \Verb|multicol| --- Multiple column environments
\end{itemize}

\paragraph{Floats and Captions}
\begin{itemize}
    \item \Verb|float| --- Improved float control with H placement specifier
    \item \Verb|caption| --- Customizable caption formatting
    \item \Verb|subcaption| --- Support for subfigures and subtables
\end{itemize}

\paragraph{Algorithms}
\begin{itemize}
    \item \Verb|algorithm| --- Floating environment for algorithms
    \item \Verb|algpseudocode| --- Pseudocode formatting commands
\end{itemize}

\paragraph{Graphics and Plotting}
\begin{itemize}
    \item \Verb|graphicx| --- Enhanced graphics inclusion
    \item \Verb|tikz| --- Powerful graphics and diagram creation
    \item \Verb|pgfplots| --- High-quality function and data plots
    \item \Verb|multimedia| --- Multimedia content embedding
\end{itemize}

\paragraph{Additional Utilities}
\begin{itemize}
    \item \Verb|pbox| --- Fixed-width paragraph boxes
    \item \Verb|nameref| --- Reference by section names
\end{itemize}

\paragraph{Document Structure}
\begin{itemize}
    \item \Verb|appendix| --- Appendix formatting and management
\end{itemize}

\paragraph{Bibliography}
\begin{itemize}
    \item \Verb|biblatex| --- Advanced bibliography management with Biber backend
\end{itemize}

\paragraph{Theorem Environments}
\begin{itemize}
    \item \Verb|thmtools| --- Enhanced theorem configuration (when \texttt{theorems=true})
    \item \Verb|tcolorbox| --- Colored and framed boxes for theorems (when \texttt{theorems=true})
\end{itemize}

\subsection{Metadata Commands}

The package provides enhanced metadata commands with optional short forms:

\begin{description}
    \item[\texttt{\textbackslash{}author[short]\{full\}}] Define document author. The optional \texttt{short} form is used in headers.
    
    \item[\texttt{\textbackslash{}title[short]\{full\}}] Define document title. The optional \texttt{short} form is used in headers.
    
    \item[\texttt{\textbackslash{}purpose\{text\}}] Specify the document's purpose (displayed in formal header).
    
    \item[\texttt{\textbackslash{}institute\{text\}}] Specify the institution or affiliation (displayed in formal header).
\end{description}

Example:
\begin{latexcode}
\title[Short Title]{Full Document Title}
\author[J. Doe]{John Doe}
\purpose{Technical Report}
\institute{Department of Computer Science}
\end{latexcode}

\subsection{Header \& Footer}

The package uses the \Verb|fancyhdr| package to create professional headers and footers.
Two header styles are available, selectable via the \texttt{header} option:

\paragraph{Default Header Style}
The default style is simple and clean:
\begin{itemize}
    \item Header: Section title centered
    \item Footer: Page number centered
    \item Header height: 0.75 inches
\end{itemize}

\paragraph{Formal Header Style}
The formal style provides comprehensive document information:
\begin{itemize}
    \item Left header (top line): Purpose and short title with author
    \item Left header (bottom line): Institute and page count (``page X of Y'')
    \item No header rule
    \item Header height: 0.75 inches
\end{itemize}

Page margins are configured as 0.5 inches on left and right sides, with 1.5 inches at the bottom.
Empty pages (in two-sided printing) are automatically cleaned using the \Verb|emptypage| package.

Footnotes should be used sparingly and are created using the \Verb|\footnote{text}| command. 
The footnote text is automatically numbered, formatted and placed at the bottom of the page.\footnote{The footnote text is automatically numbered, formatted and placed at the bottom of the page like this.}


\subsection{Document Structure}

The package provides commands for modular document organization:

\begin{description}
    \item[\texttt{\textbackslash{}setdirectory[path]}] Set the base directory for content files (default: current directory).
    
    \item[\texttt{\textbackslash{}setappendix[path]}] Switch to appendix mode and set the directory for appendix files.
    
    \item[\texttt{\textbackslash{}addcontent\{filename\}}] Include content from the specified file in the current directory.
\end{description}

Example usage:
\begin{latexcode}
% Include main content
\setdirectory[content]
\addcontent{1_introduction}
\addcontent{2_methodology}

% Include appendices
\setappendix[appendix]
\addcontent{1_proofs}
\addcontent{2_data}
\end{latexcode}

The default document class is \Verb|article| with 11pt font size and two-sided printing.
Sections are numbered automatically with hierarchical structure: section, subsection, and subsubsection.

\paragraph{Section} 
Create a new section with \Verb|\section{Section title}|.
Within sections, use \Verb|\subsection{Subsection title}| for subsections and \Verb|\subsubsection{Subsubsection title}| for deeper nesting.

\paragraph{Paragraph} 
Create a titled paragraph with \Verb|\paragraph{Paragraph title}|.
The paragraph title appears in boldface, with text following on the same line.
To start a new paragraph without a title, simply leave a blank line.

\subsection{Hyperlink \& Reference}

The \href{https://ctan.org/pkg/hyperref}{\Verb|hyperref|} package creates clickable hyperlinks in the PDF output.
All links use the default text color (no colored boxes) for print-friendly documents.

\paragraph{External Links}
Create hyperlinks with \Verb|\href{url}{text}|, where \texttt{url} is the web address and \texttt{text} is the displayed link text.

\paragraph{Internal References}
The package provides convenient reference commands for different element types:

\textbf{Standard references:}
\begin{itemize}
    \item \Verb|\ref{label}| --- Basic reference to labeled item
    \item \Verb|\nameref{label}| --- Reference by section name
    \item \Verb|\pageref{label}| --- Reference to page number
\end{itemize}

\textbf{Specialized reference commands:}
\begin{itemize}
    \item \Verb|\refalg{name}| --- Reference to Algorithm~\ref{alg:name}
    \item \Verb|\reffig{name}| --- Reference to Figure~\ref{fig:name}
    \item \Verb|\reftab{name}| --- Reference to Table~\ref{tab:name}
\end{itemize}

\textbf{Corresponding label commands:}
\begin{itemize}
    \item \Verb|\labelalg{name}| --- Equivalent to \Verb|\label{alg:name}|
    \item \Verb|\labelfig{name}| --- Equivalent to \Verb|\label{fig:name}|
    \item \Verb|\labeltab{name}| --- Equivalent to \Verb|\label{tab:name}|
\end{itemize}

When the \texttt{theorems} option is enabled, additional reference commands are available for theorem-like environments (see Section~\ref{sec:theorem-refs}).

\subsection{Citation \& Bibliography}

The \href{https://ctan.org/pkg/biblatex}{\Verb|biblatex|} package manages citations and bibliographies with the Biber backend.
Biber provides superior Unicode support, advanced sorting, and better handling of complex bibliography requirements compared to BibTeX.

\paragraph{Setup}
Specify the bibliography file in the preamble:
\begin{latexcode}
\addbibresource{references.bib}
\end{latexcode}

\paragraph{Citations}
Create citations using:
\begin{itemize}
    \item \Verb|\autocite{key}| --- Context-sensitive citation (recommended)
    \item \Verb|\cite{key}| --- Standard citation
    \item \Verb|\textcite{key}| --- Textual citation (e.g., ``Author (2023) shows...'')
    \item \Verb|\parencite{key}| --- Parenthetical citation
\end{itemize}

Citations appear with square brackets (e.g., [1]) regardless of the bibliography style.

\paragraph{Bibliography}
Print the bibliography at the document end:
\begin{latexcode}
\printbibliography
\end{latexcode}

The bibliography style is set via the \texttt{bibstyle} package option (default: \texttt{ieee}).
Common styles include \texttt{numeric}, \texttt{alphabetic}, \texttt{ieee}, \texttt{apa}, and \texttt{nature}.