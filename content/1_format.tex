\section{Format}

This is a template for creating \LaTeX{} documents. 
The style file \Verb|document.sty|, written in \LaTeXe{}, defines the overall layout and formatting of the document.
The theorem-like environments are inspired by the template~\autocite{sleepymalc2022template}.
For detail implementation, refer to the GitHub repository~\autocite{chang2025document}.

\subsection{Section \& content}

The default font size is set to 11pt, and the document class is set to \Verb|article| with two-sided printing.
The document is divided into sections, and the content is included from separate files using the \Verb|\addcontent| command.
The sections are numbered automatically, and the section titles are displayed in boldface.
The content is organized into directories, with the main content in the \Verb|content| directory and the appendix content in the \Verb|appendix| directory.
The \Verb|\setdirectory| command is used to specify the directory for the main content, and the \Verb|\setappendix| command is used to specify the directory for the appendix content.
The \Verb|\addcontent| command is used to include the content from the specified directory.
To add a new section, create a new file in the \Verb|content| directory and use the \Verb|\addcontent{filename}| command to include it in the main document.
In the new file, use the \Verb|\section{Section Title}| command to create a new section.
Within the section, you can use the \Verb|\subsection{Subsection title}| command to create subsections.

\subsection{Header \& footer}

The header and footer are set using the \Verb|fancyhdr| package.
The header contains the section number and title at the center, and the footer contains the page number also at the center.
The header height is set to 0.5 inches and the margins are set to 0.5 inches on the left and right, and 1 inch at the bottom.
Footnotes are provided and should be used sparingly. 
If you do require a footnote, it should be properly typeset after the phrase (and after punctuation marks if there is one.)
Footnotes are created using the \Verb|\footnote{text}| command. 
The footnote text is placed at the bottom of the page.\footnote{The footnote text is automatically numbered, formatted and placed at the bottom of the page like this.}


\subsection{Hyperlink \& reference}

The \href{https://ctan.org/pkg/hyperref}{\Verb|hyperref|} package is used to create hyperlinks in the document.
The color box is disabled for the links to ensure that the document is print-friendly.
A sample usage is as follows: \Verb|\href{link}{text}|.
Other than the hyperlinks, the other is using the macro \Verb|\ref{name}| to reference the labeled (sub)section, equation, figure, table, algorithm, or theorem-like environment.
The labeling is done using the \Verb|\label{name}| command, which is placed right after the (sub)section or within the environment.

\subsection{Citation \& bibliography}

The \href{https://ctan.org/pkg/biblatex}{\Verb|biblatex|} package is used for managing citations and bibliographies.
The bibliography file is specified using the \Verb|\addbibresource{filename.bib}| command.
Citations are created using the \Verb|\autocite{key}| command, where \Verb|key| is the citation key defined in the bibliography file.
The bibliography is printed at the end of the document using the \Verb|\printbibliography| command.
The bibliography style is set to APA using the \Verb|style=apa| option when loading the \Verb|biblatex| package.
The backend for the bibliography is set to Biber using the \Verb|backend=biber| option.
While the bibtex is more classical, Biber provides more advanced features and better support for modern bibliography styles.
It is recommended to use Biber for new documents, and this template is set up to use Biber by default.