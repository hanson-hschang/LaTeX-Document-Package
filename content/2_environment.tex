\section{Environment}

An \emph{environment} in {\LaTeX} is a section of the document that is formatted in a specific way. 
This document demonstrates various environments, including the use of scripts, tables, figures, algorithms, and framed environments.
The basic usage of an environment is shown below.

\begin{minted}{latex}
\begin{environment}
  some content included in the environment
\end{environment}
\end{minted}

Here are some common environments used in {\LaTeX} including \texttt{enumerate}, \texttt{itemize}, and \texttt{displayquote}.

\noindent\rule{\linewidth}{0.4pt}

\begin{multicols}{2}
  \begin{enumerate}
    \item First item in an enumerated list.
    \item Second item in the same list.
    \item Third item and more ...
  \end{enumerate}
  \begin{latexcode}
\begin{enumerate}
  \item First item in an enumerated list.
  \item Second item in the same list.
  \item Third item and more ...
\end{enumerate}
  \end{latexcode}
\end{multicols}

\noindent\rule{\linewidth}{0.4pt}

\begin{multicols}{2}
  \begin{itemize}
    \item First item in an unordered list.
    \item Second item in the same list.
    \item Third item and more ...
  \end{itemize}
  \begin{latexcode}
\begin{itemize}
  \item First item in an unordered list.
  \item Second item in the same list.
  \item Third item and more ...
\end{itemize}
  \end{latexcode}
\end{multicols}

\noindent\rule{\linewidth}{0.4pt}

\begin{multicols}{2}
  \begin{displayquote}
    This is a quote, which uses the \texttt{csquotes} package with customization set inside the \texttt{document.sty} file.
  \end{displayquote}
  \columnbreak{}
  \begin{latexcode}
\begin{displayquote}
  This is a quote, which uses the \texttt{csquotes} package with customization set inside the \texttt{document.sty} file.
\end{displayquote}
  \end{latexcode}
\end{multicols}

\noindent\rule{\linewidth}{0.4pt}

\subsection{Scripts}

To include scripts in your document, the \texttt{minted} package is used.
Here is an example of a \texttt{python} script:

\begin{multicols}{2}
  \begin{minted}{python}

def hello_world():
    print("Hello, World!")
hello_world()
  \end{minted}
  \columnbreak{}
  \begin{latexcode}
\begin{minted}{python}
def hello_world():
    print("Hello, World!")
hello_world()
\end{minted}
  \end{latexcode}
\end{multicols}

\subsection{Tables}
Use the package \texttt{booktabs} for better table formatting.

\subsection{Figures}

\subsection{Algorithms}

\subsection{Frames}

This document demonstrates the usage of the custom theorem-like environments defined in the \texttt{document.sty} package.

\begin{definition}[Sample Definition]
  This is an example of a definition environment.
\end{definition}

\begin{assumption}[Sample Assumption]
  This is an example of an assumption environment.
\end{assumption}

\begin{conjecture}[Sample Conjecture]
  This is an example of a conjecture environment.
\end{conjecture}

\begin{lemma}[Sample Lemma]
  This is an example of a lemma. It shares the same style as the theorem environment.
\end{lemma}
\begin{proof}
  This is an example of a proof environment. It is used to provide a proof for the lemma.
\end{proof}

\begin{theorem}[Sample Theorem]
  This is an example of a theorem environment. 
\end{theorem}
\begin{proof}
  This is an example of a proof for the theorem.
\end{proof}

\begin{corollary}[Sample Corollary]
  This is an example of a corollary. It also shares the same style as the theorem environment.
\end{corollary}
\begin{proof}
  This is an example of a proof for the corollary.
\end{proof}

\begin{proposition}[Sample Proposition]
  This is an example of a proposition. It also shares the same style as the theorem environment.
\end{proposition}
\begin{proof}
  This is an example of a proof for the proposition.
\end{proof}

\begin{remark}[Sample Remark]
  This is an example of a remark environment.
\end{remark}

\begin{example}[Sample Example]
  This is an example of an example environment.
\end{example}

\begin{solution}[Sample Answer]
  This is an example of an solution environment.
\end{solution}

\begin{problem}[Sample Problem]
  This is an example of a problem environment.
\end{problem}

These environments are designed to enhance the readability and organization of mathematical documents. 
Each environment has a specific purpose and style, making it easier for readers to follow the logical flow of the content.