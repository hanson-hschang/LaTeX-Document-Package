\section{Environment}

\subsection{Tables}
Use the package \texttt{booktabs} for better table formatting.

\subsection{Figures}

\subsection{Algorithms}

\subsection{Theorem-like Environments}

This document demonstrates the usage of the custom theorem-like environments defined in the \texttt{document.sty} package.

\begin{definition}[Sample Definition]
  This is an example of a definition environment.
\end{definition}

\begin{assumption}[Sample Assumption]
  This is an example of an assumption environment.
\end{assumption}

\begin{conjecture}[Sample Conjecture]
  This is an example of a conjecture environment.
\end{conjecture}

\begin{lemma}[Sample Lemma]
  This is an example of a lemma. It shares the same style as the theorem environment.
\end{lemma}
\begin{proof}
  This is an example of a proof environment. It is used to provide a proof for the lemma.
\end{proof}

\begin{theorem}[Sample Theorem]
  This is an example of a theorem environment. 
\end{theorem}
\begin{proof}
  This is an example of a proof for the theorem.
\end{proof}

\begin{corollary}[Sample Corollary]
  This is an example of a corollary. It also shares the same style as the theorem environment.
\end{corollary}
\begin{proof}
  This is an example of a proof for the corollary.
\end{proof}

\begin{proposition}[Sample Proposition]
  This is an example of a proposition. It also shares the same style as the theorem environment.
\end{proposition}
\begin{proof}
  This is an example of a proof for the proposition.
\end{proof}

\begin{remark}[Sample Remark]
  This is an example of a remark environment.
\end{remark}

\begin{example}[Sample Example]
  This is an example of an example environment.
\end{example}

\begin{solution}[Sample Answer]
  This is an example of an solution environment.
\end{solution}

\begin{problem}[Sample Problem]
  This is an example of a problem environment.
\end{problem}

These environments are designed to enhance the readability and organization of mathematical documents. 
Each environment has a specific purpose and style, making it easier for readers to follow the logical flow of the content.