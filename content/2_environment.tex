\section{Environment}

An \emph{environment} in \LaTeX{} is a section of the document formatted in a specific way. 
This document demonstrates various environments, including lists, quotes, code blocks, scripts, tables, figures, algorithms, and framed theorem-like environments.
The basic syntax of an environment is:

\begin{latexcode}
\begin{environment}
  content included in the environment
\end{environment}
\end{latexcode}

\subsection{Lists and Quotes}

Common text environments include \Verb|enumerate| (numbered lists), \Verb|itemize| (bulleted lists), and \Verb|displayquote| (block quotations).
These environments can be customized using the \Verb|enumitem| package for lists and \Verb|csquotes| package for quotations.

\noindent\rule{\linewidth}{0.5pt}

\begin{multicols}{2}
  \begin{enumerate}
    \item First item in an enumerated list.
    \item Second item in the same list.
    \item Third item and more ...
  \end{enumerate}
  \begin{latexcode}
\begin{enumerate}
  \item First item in an enumerated list.
  \item Second item in the same list.
  \item Third item and more ...
\end{enumerate}
  \end{latexcode}
\end{multicols}

\noindent\rule{\linewidth}{0.5pt}

\begin{multicols}{2}
  \begin{itemize}
    \item First item in an unordered list.
    \item Second item in the same list.
    \item Third item and more ...
  \end{itemize}
  \begin{latexcode}
\begin{itemize}
  \item First item in an unordered list.
  \item Second item in the same list.
  \item Third item and more ...
\end{itemize}
  \end{latexcode}
\end{multicols}

\noindent\rule{\linewidth}{0.5pt}

\begin{multicols}{2}
  \begin{displayquote}
    This is a quote, which uses the \verb|csquotes| package with customization set inside the \verb|document.sty| file.
  \end{displayquote}
  \columnbreak{}
  \begin{latexcode}
\begin{displayquote}
  This is a quote, which uses the \verb|csquotes| package with customization set inside the \verb|document.sty| file.
\end{displayquote}
  \end{latexcode}
\end{multicols}

\noindent\rule{\linewidth}{0.5pt}
\vfill

\pagebreak

\subsection{Script}

\paragraph{Inline Code}
Use the \Verb|fancyvrb| package's \Verb|\Verb| command for inline code.
For example, \Verb+\Verb|print("Hello, World!")|+ produces \Verb|print("Hello, World!")|.
The delimiter (typically \texttt{|}) can be any character not appearing in the code.

\paragraph{Code Blocks}
The \Verb|minted| package provides syntax-highlighted code blocks for various programming languages.
When the \texttt{code=true} option is enabled, the package pre-configures \Verb|latexcode| and \Verb|pythoncode| environments.

\textbf{Python example:}

\begin{multicols}{2}
  \begin{pythoncode}
def fibonacci(n):
    """Calculate nth Fibonacci number."""
    if n <= 1:
        return n
    return fibonacci(n-1) + fibonacci(n-2)

# Compute first 10 Fibonacci numbers
for i in range(10):
    print(f"F({i}) = {fibonacci(i)}")
  \end{pythoncode}

  \columnbreak{}
  
  \begin{latexcode}
\begin{pythoncode}
def fibonacci(n):
    """Calculate nth Fibonacci number."""
    if n <= 1:
        return n
    return fibonacci(n-1) + fibonacci(n-2)

# Compute first 10 Fibonacci numbers
for i in range(10):
    print(f"F({i}) = {fibonacci(i)}")
\end{pythoncode}
  \end{latexcode}
\end{multicols}

The \Verb|minted| package supports many languages including Python, C, C++, Java, JavaScript, and more.
Custom language environments can be added using the \Verb|\newminted| command (see lines 211-235 in \Verb|document.sty| for examples of \Verb|latexcode| and \Verb|pythoncode| configurations).


\subsection{Mathematics}

The package includes comprehensive mathematics support through AMS packages and specialized math fonts.

\paragraph{Inline and Display Math}
Inline math: $E = mc^2$ is typeset with \Verb|$E = mc^2$|.
Display math uses \Verb|\[...\]| or \Verb|equation| environment:
\[
  \int_{-\infty}^{\infty} e^{-x^2} dx = \sqrt{\pi}
\]

\paragraph{Special Symbols and Fonts}
The package provides extensive symbol support:
\begin{itemize}
    \item Blackboard bold: $\mathbb{R}, \mathbb{C}, \mathbb{N}, \mathbb{Z}, \mathbb{Q}$
    \item Doublestroke: $\mathds{1}, \mathds{R}$
    \item Calligraphic: $\mathcal{L}, \mathcal{F}, \mathcal{H}$
    \item Script: $\mathscr{L}, \mathscr{F}, \mathscr{H}$
    \item Fraktur: $\mathfrak{g}, \mathfrak{h}, \mathfrak{su}$
    \item Upright Greek: $\uppi, \upmu, \upsigma$
\end{itemize}

\paragraph{Fractions and Units}
Nice diagonal fractions: $\nicefrac{1}{2}, \nicefrac{a}{b}$ using \Verb|\nicefrac{1}{2}|.
SI units with \Verb|siunitx|: \SI{299792458}{\meter\per\second} (speed of light), \SI{1.602e-19}{\coulomb} (elementary charge), \SI{6.626e-34}{\joule\second} (Planck constant).

\paragraph{Systems of Equations}
The \Verb|systeme| package formats systems elegantly:
\[
  \systeme{
    x + y + z = 6,
    2x - y + z = 3,
    x + 2y - z = 2
  }
\]

\subsection{Algorithm}

Algorithms should be presented clearly and concisely, typically placed at the top of a page using the \Verb|[t]| placement specifier.
The \Verb|algorithm| package provides the floating environment, while \href{https://www.ctan.org/pkg/algpseudocodex/}{\Verb|algpseudocode|} provides pseudocode formatting commands.
See \refalg{sample} for an example.

\addcontent{sample_algorithm}
  
\subsection{Table}

Tables should be centered, legible, and positioned at the top of the page.
Table captions appear \emph{before} the table content and use sentence case (capitalize only the first word and proper nouns).
Publication-quality tables avoid vertical rules~\autocite{neurips2023template}. 
The \href{https://www.ctan.org/pkg/booktabs}{\Verb|booktabs|} package provides professional horizontal rules: \Verb|\toprule|, \Verb|\midrule|, \Verb|\cmidrule|, and \Verb|\bottomrule|.
Additional packages support dashed lines (\Verb|arydshln|), multi-row cells (\Verb|multirow|), and multi-column layouts (\Verb|multicol|).
See \reftab{sample} for an example.

\addcontent{sample_table}

\subsection{Figure}

Figures must be centered, legible, and positioned at the top of the page. 
Figure captions appear \emph{after} the figure content, using sentence case and a smaller font than the main text. 
The \Verb|caption| and \Verb|subcaption| packages enable sophisticated caption formatting including subfigures.
The package supports multiple graphics formats via \Verb|graphicx| and provides TikZ/PGFPlots for creating publication-quality diagrams and plots.
Colored figures are acceptable when they remain legible in both color and grayscale printing.
See \reffig{sample} for examples.

\addcontent{sample_figure}

\subsection{Frame}
\label{sec:theorem-refs}

When the \texttt{theorems=true} option is enabled, the package provides elegant theorem-like environments using \Verb|tcolorbox| with color-coded frames.
All environments share the same counter (numbered within subsections) and support three arguments: title, label, and content.

\begin{definition}{Continuity}{continuity}
  A function $f: \mathbb{R} \to \mathbb{R}$ is \emph{continuous} at point $x_0$ if for every $\epsilon > 0$, there exists $\delta > 0$ such that $|x - x_0| < \delta$ implies $|f(x) - f(x_0)| < \epsilon$.
\end{definition}

\begin{assumption}{Bounded Domain}{bounded}
  We assume the domain $\Omega \subset \mathbb{R}^n$ is bounded, connected, and has a piecewise smooth boundary $\partial\Omega$.
\end{assumption}

\begin{conjecture}{Collatz Conjecture}{collatz}
  For any positive integer $n$, the iterative sequence defined by $a_{i+1} = a_i/2$ if $a_i$ is even, or $a_{i+1} = 3a_i + 1$ if $a_i$ is odd, eventually reaches 1.
\end{conjecture}

\begin{lemma}{Triangle Inequality}{triangle}
  Based on \refdef{continuity}, for any $x, y \in \mathbb{R}^n$, we have $\|x + y\| \leq \|x\| + \|y\|$, where $\|\cdot\|$ denotes the Euclidean norm.
\end{lemma}
\begin{proof}
  By the Cauchy-Schwarz inequality, $\langle x, y \rangle \leq \|x\|\|y\|$. Then:
  \[
    \|x + y\|^2 = \|x\|^2 + 2\langle x, y \rangle + \|y\|^2 \leq \|x\|^2 + 2\|x\|\|y\| + \|y\|^2 = (\|x\| + \|y\|)^2
  \]
  Taking square roots of both sides yields the result.
\end{proof}

\begin{theorem}{Fundamental Theorem of Calculus}{ftc}
  If $f$ is continuous on $[a, b]$ and $F$ is an antiderivative of $f$ on $[a, b]$, then $\int_a^b f(x)\,dx = F(b) - F(a)$.
\end{theorem}
\begin{proof}[Proof sketch]
  Define $G(x) = \int_a^x f(t)\,dt$. By continuity of $f$, we have $G'(x) = f(x)$. Since both $F$ and $G$ are antiderivatives of $f$, they differ by a constant: $F(x) = G(x) + C$. Evaluating at $x = a$ gives $C = F(a)$, and at $x = b$ gives the result.
\end{proof}

\begin{corollary}{Integration by Parts}{parts}
  If $u$ and $v$ are continuously differentiable functions, then $\int u\,dv = uv - \int v\,du$.
\end{corollary}
\begin{proof}
  This follows directly from \refthm{ftc} applied to the product rule: $(uv)' = u'v + uv'$.
\end{proof}

\begin{proposition}{Derivative of Exponential}{exp-deriv}
  For the exponential function $e^x$, we have $\frac{d}{dx}e^x = e^x$.
\end{proposition}
\begin{proof}
  Using the limit definition:
  \[
    \frac{d}{dx}e^x = \lim_{h \to 0} \frac{e^{x+h} - e^x}{h} = e^x \lim_{h \to 0} \frac{e^h - 1}{h} = e^x
  \]
\end{proof}

\begin{remark}{Uniqueness of Exponential}{exp-unique}
  The exponential function is the only function (up to scaling) that is its own derivative, making it fundamental in differential equations.
\end{remark}

\begin{note}{Notation Convention}{notation}
  Throughout this document, we use $\mathbb{N}$ for natural numbers, $\mathbb{Z}$ for integers, $\mathbb{Q}$ for rationals, $\mathbb{R}$ for reals, and $\mathbb{C}$ for complex numbers.
\end{note}

\begin{example}{Computing a Derivative}{derivative-ex}
  Find the derivative of $f(x) = x^2 \sin(x)$ using the product rule.
\end{example}
\begin{solution}
  By the product rule: $f'(x) = (x^2)'\sin(x) + x^2(\sin(x))' = 2x\sin(x) + x^2\cos(x)$.
\end{solution}

\begin{problem}{Optimization Problem}{optimization}
  A rectangular box with square base must have volume \SI{1000}{\cubic\centi\meter}. Find the dimensions that minimize the surface area.
\end{problem}

\begin{todo}{Future Enhancement}{future}
  Extend the package to support additional theorem styles and custom color schemes for different document themes.
\end{todo}

These environments are designed to enhance the readability and organization of mathematical documents. 
Each environment has a specific purpose and style, making it easier for readers to follow the logical flow of the content.

\paragraph{Reference Commands}
The package provides convenient reference commands for theorem-like environments:
\begin{itemize}
    \item \Verb|\refdef{label}| --- Reference to Definition (equivalent to \texttt{\textbackslash{}ref\{def:label\}})
    \item \Verb|\refasm{label}| --- Reference to Assumption (equivalent to \texttt{\textbackslash{}ref\{asm:label\}})
    \item \Verb|\refconj{label}| --- Reference to Conjecture (equivalent to \texttt{\textbackslash{}ref\{conj:label\}})
    \item \Verb|\reflem{label}| --- Reference to Lemma (equivalent to \texttt{\textbackslash{}ref\{lem:label\}})
    \item \Verb|\refthm{label}| --- Reference to Theorem (equivalent to \texttt{\textbackslash{}ref\{thm:label\}})
    \item \Verb|\refcor{label}| --- Reference to Corollary (equivalent to \texttt{\textbackslash{}ref\{cor:label\}})
    \item \Verb|\refprop{label}| --- Reference to Proposition (equivalent to \texttt{\textbackslash{}ref\{prop:label\}})
    \item \Verb|\refremk{label}| --- Reference to Remark (equivalent to \texttt{\textbackslash{}ref\{remk:label\}})
    \item \Verb|\refnote{label}| --- Reference to Note (equivalent to \texttt{\textbackslash{}ref\{note:label\}})
    \item \Verb|\refex{label}| --- Reference to Example (equivalent to \texttt{\textbackslash{}ref\{ex:label\}})
    \item \Verb|\refprob{label}| --- Reference to Problem (equivalent to \texttt{\textbackslash{}ref\{prob:label\}})
    \item \Verb|\reftodo{label}| --- Reference to ToDo (equivalent to \texttt{\textbackslash{}ref\{todo:label\}})
\end{itemize}

These commands automatically include the environment type (e.g., ``Definition'') and number, creating clear cross-references throughout the document.